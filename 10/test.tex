\documentclass{article}

\usepackage[UTF8]{ctex}

% 使用\newcommand可以是简单的字符串替换
\newcommand\PRC{People's Republic of \emph{China}}

% \newcommand可以使用参数
% 参数个数可以从 1 到 9,使用时用 #1, #2, ..., #9表示
\newcommand\loves[2]{#1 喜欢 #2}
\newcommand\hateby[2]{#2 不受 #1 喜欢}

% \newcommand的参数可以指定默认值
% 指定参数个数的同时制定了首个参数的默认值,那么这个命令的
% 第一个参数就成为可选的参数(要使用中括号指定)
\newcommand{\love}[3][喜欢]{#2#1#3}

% \renewcommand-重定义命令
% 只能作用于已有的命令
\renewcommand{\abstractname}{内容简介}

% 定义和重定义环境
% \newenvironment
% \renewenvironment
\newenvironment{myabstract}[1][我的摘要]
{
    \begin{center}\textbf #1\end{center}
    \begin{quote} 
}
{\end{quote}}


% 环境参数只有环境前定义中可以使用参数
% 可以和newcommand嵌套使用
\newenvironment{Quotation}[1]
{
    \newcommand\quotesource{#1}
    \begin{quotation}
}
{\par\hfill--- 《\textit{\quotesource}》
\end{quotation}}

% 正文区
\begin{document}
    \PRC

    \loves{猫儿}{鱼}

    \hateby{猫儿}{萝卜}

    \love[最爱]{猫儿}{鱼}

    \love{猫儿}{鱼}

    \begin{abstract}
        这是一段摘要
    \end{abstract}

    \begin{myabstract}
        这是一段自定义格式的摘要
        \par\hfill--- 《\textit{乾坤}》
    \end{myabstract}

    \begin{Quotation}{易$\cdot$乾}
        初九,潜龙勿用。
    \end{Quotation}

\end{document}