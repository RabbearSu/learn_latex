\documentclass{article}

\usepackage{ctex}

\usepackage{amsmath}

\begin{document}
    \section{简介}
    \LaTeX 的排版模式分为文本模式和数学模式
    \section{行内公式}
    \subsection{美元符号}
    交换律是$a+b=b+a$, 如$1+2=2+1=3$。
    \subsection{小括号}
    交换律是\(a+b=b+a\)
    \subsection{math环境}
    交换律是\begin{math}a+b=b+a\end{math},如\begin{math}1+2=2+1=3\end{math}
    \section{上下标}
    \subsection{上标}
    $$3x^{3x^{20}} - 3x + 4 = 6$$
    \subsection{下标}
    $a_0, a_1, a_2$, \(tuxiong_0, tuxiong_1\)
    \begin{math}
        a_{3x^{20} + 3x + 5}
    \end{math}
    \section{希腊字母}
    $\alpha$
    $\beta$
    $\gamma$
    $\epsilon$
    $\pi$
    $\omega$

    $\Gamma$

    $$\alpha^3 + \beta^4 = \gamma^2$$
    \section{数学函数}
    $\log$
    $\sin$
    $\cos$
    $\arccos$
    $\arcsin$

    $$\sin^2 x + \cos^2 x = 1$$

    $y = \sin^{-1} x$

    $y = \log_2 x$

    $y = \ln x$

    $\sqrt{2}$
    $\sqrt{x^2 + y^2}$
    $\sqrt{2 + \sqrt{2}}$
    $\sqrt[5]{x}$
    \section{分式}
    大约是原体积的$3/4$
    
    大约是原体积的$\frac{3}{4}$

    $$\frac{1}{1+\frac{1}{x}}$$

    $\frac{\sqrt{x+1}}{\sqrt{x-1}}$
    $\sqrt{\frac{x}{x^2 + 3x -5}}$
    \section{行间公式}
    \subsection{美元符号}
    $$a + v = 4$$
    \subsection{中括号}
    交换律是\[a + b = b + a\]
    \subsection{displaymath环境}
    \begin{displaymath}
        a + b = b + a
    \end{displaymath}
    \subsection{自动编号}
    交换律见公式\ref{eq:1}
    % \begin{equation}
    %     a + b = b + a \label{eq:1}
    % \end{equation}
    \subsection{不编号公式equation*}
    交换律公式见式\ref{eq:2}
    \begin{equation*}
        a + b = b + a \label{eq:2}
    \end{equation*}

    再如公式\ref{eq:pol}:
    \begin{equation}
        3x^2 + 5x -6 = 8 \label{eq:pol}
    \end{equation}
\end{document}